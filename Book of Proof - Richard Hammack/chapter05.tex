\documentclass[main.tex]{subfiles}
\begin{document}

\section{Prova Contrapositiva}
As provas diretas partem de uma proposição $P$ para a conclusão $Q$ mostrando sem desvios, porém agora aprenderemos uma nova técnica, onde supomos de $\sim Q$ e concluímos $\sim P$.
\\~\\
Pela tabela verdade fica fácil ver que $P \Rightarrow Q$ e $ \sim Q \Rightarrow \sim P$ são \textbf{lógicamente equivalentes}.
\\~\\
Se escrever uma tabela verdade ficará fácil de ver.
\\~\\
Vamos fazer alguns exemplos para simplificar a compreensão
\begin{proposition}
Se $n \ \in  \mathbb{Z}$ e $n^2$ é par, então $n$ é par.
\end{proposition}
\textbf{Prova:} Suponha que $n$ seja impar, ou seja, $n = 2a+1, a \in \mathbb{I}$, então \\ $n^2 = (2a +1)^2 = 4a^2 +2a + 1 = 2(2a^2 + a) + 1 \ \therefore \ n^2$ é impar.
\begin{proposition}
Suponha $a, \ b \in \mathbb{Z}$. Se $a^2(b^2 -2b)$ é impar, então $a, \ b$ são impares.
\end{proposition}
\textbf{Prova:} Suponha $a, \ b$ pares tq. $a = 2x$ e $2y$, então:\\
$a^2(b^2 -2b) = 4x^2(4y^2 - 4y) = 16x^2y^2 -16yx^2 = 2(8x^2y^2 -16x^2y)$.\\
Logo, $a^2(b^2 -2b)$ é par

\section{Congruência dos Inteiros}
\begin{definition}[n-Módulo Congruente]
Dados $a, \ b $ e $n \in \mathbb{N}$, dizemos que são n-módulo congruentes de $n | (a-b)$. Expressamos como $ a \equiv b \ mod(n)$. Se $a$ e $b$ não são n-módulo congruentes expressamos como $a \not\equiv  b \ mod(n)$
\end{definition}
O que isso nos diz é que se $a \equiv b \ mod(n)$, então o resto da divisão de $a/n$ e $b/n$ são iguais. \\~\\
Isso é facil de mostrar, se $a = xn + r$ e $b = yn + r$, então $a-b = n(x-y)$ como podemos dividir ambos os lados da equação por $n$ e ainda ter valores inteiros, então $a \equiv b \ mod(n)$.
\section{A Escrita Matemática}
Uma observação importante a ser feita antes de começar a leitura dessa sessão é que praticamente apenas traduzimos os enunciados das "normas", pois não vimos boa maneira para resumir. Mas ainda assim tornamos o conteúdo acessível e respeitando o trabalho de Richard Hammack. \\~\\ 
Ao escrever provas temos que nos atentar tanto ao rigor matemático quanto à clareza, para isso existem boas práticas para a escrita matemática que recomendamos que sejam seguidas. São elas:
\\~\\
\textbf{01. Comece as frases com palavras, não símbolos matemática}
\\~\\
\textbf{02. Termine as sentenças com pontos} (mesmo que terminem com sibolos matemáticos).
\\~\\
\textbf{03. Separe símbolos e expressões com palavras.}
\\~\\
\textbf{04. Evite mal-uso de símbolos.}
\\~\\
\textbf{05. Evite uso desnecessário de símbolos.}
\\~\\
\textbf{06. Faça uso da primeira pessoa do plural.}
\\~\\
\textbf{07. Use voz ativa.}
\\~\\
\textbf{08. Explique cada novo símbolo.}
\\~\\
\textbf{09. Cuidado com os "isso".}
\\~\\
\textbf{10. "Desde", "porque", "como", "para", "então"}, são palavras muito comuns em provas, porém num geral significam a mesma coisa quando utilizamos a forma condicional $Q \\Rightarrow P$.
\\~\\
\textbf{11. "Assim", "portanto", "logo", "consequentemente}, outras palavras que aparecem muito frequentemente nas provas, porém são mais comum nos finais.
\\~\\
\textbf{12. Clareza é o padrão-ouro da escrita matemática.}
%%%%%%%%%%%%%%%%%%%%%%%% CHAPTER %%%%%%%%%%%%%%%%%%%%%%%%
\end{document}