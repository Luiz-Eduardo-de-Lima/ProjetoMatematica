\documentclass[main.tex]{subfiles}
\begin{document}

Quando estudamos matemática é muito comum nos depararmos com provas, saber lê-las é muito importante para a compreenção de princípios matemática e teoremas,. Tão importante quanto saber ler uma prova é saber fazê-la, o processo de provar afirmações mostra que, além de ter entendido, você consegue transmitir esse conhecimento, com o tempo e prática conseguirá fazer boas demonstrações.
\\~\\
Como o objetivo do curso é um preparatório para uma matemática bem diferente do que vemos durante o Ensino Médio (ou até mesmo em algumas graduações, infelizmente), fugindo da maneira mecânica e da memorização de fórmulas, é de extrema importância que essa parte não seja omitida, agora vamos entender o que são \textbf{teoremas}, \textbf{definições}, \textbf{colorários}, etc, e algumas técnicas de \textbf{demonstração}.

\section{Teoremas}
No decorrer do nosso curso já vimos alguns teoremas e suas provas, e todos seguem um padrão, são proposições \textbf{condicionais}. Relembrando um pouco de nosso estudo de lógica, as condicionais seguem o formato "Se P, então Q", boa parte dos teoremas são escritos exatamente assim, mas são facilmente compreendidos dessa maneira.
\\
Vamos usar o \textbf{Teorema Binomial} como exemplo:
\begin{theorem}[Teorema Binomial]
Seja $n \in \mathbb{N}, \ n>0$ , logo:
$$(a+b)^n = {n \choose 0}a^{n-0}b^{0} + {n \choose 1}a^{n-1}b^1 + \cdots + {n \choose n-1}a^1b^{k-1} + {n \choose n}a^{0}b^k $$
\end{theorem}
Repare que podemos trocar "Seja" por "Se" e "logo" por "então" sem que haja a perda ou distorção do significado.
\\
O valor $n$ ser um número natural e maior do que zero é a condição para que o restante do teorema seja verdadeiro, logo, é uma relação condicional. Você pode fazer esse exercício com outros teoremas, substituindo os termos utilizados e vendo se o significado altera-se.
\\
Outro caso possível são os teoremas \textbf{bicondicionais} ($\Leftrightarrow$), exemplo:
\\
\begin{theorem}
$n$ é par se, e somente se, $n^2$ é par.
\end{theorem}
Aqui temos duas condições, é fácil vermos que um é condição para o outro, por isso, bicondicional.
\\
Mas não somente teoremas seguem essas estruturas lógicas, sempre iremos nos depara com outros termos na nossa jornada pela matemática, por isso precisamos dar o significado certinho de cada um deles.
\\~\\
\textbf{Teorema:} reservado para afirmações mais importantes.
\\~\\
\textbf{Proposições:} ainda uma afirmação, mas de significância inferior.
\\~\\
\textbf{Colorário:} consequência imediata de um teorema ou proposição.
\\~\\
\textbf{Lema:} teorema com propósito de melhorar outro. 
\section{Definições}
Nessa sessão não iremos nos prolongar, é muito simples, aqui haverão algumas observações e nada que vá muito além disso.
\\~\\
É sempre importante ressaltar a importância de definições precisas, uma técnica de demonstração é a por \textbf{absurdo} onde primeiro negamos o teorema e disso chegamos à uma conclusão que contradiga uma definição.
\\~\\
As definições servem que fujamos de ambiguidades e confusões, ao ler e escrever uma prova é preciso que ambos concordem com as definições envolvidas, falar o mesmo "idioma" é obrigatório.
\\~\\
Outra coisa que precisamos ressaltar é que \textbf{não tem como provar uma definição}, definições tem apenas o papel de comunicar o que é o que, não há como demonstrar.
\section{Prova Direta}
As provas diretas são aquelas que partimos direto da condição para a conclusão, sem tentar chegar a um absurdo por exemplo.
\\~\\
Mas também não é tão simples assim, já sabemos o começo e o final do teorema ou proposição, nosso objetivo é desenvolver a prova partindo da condição obedecendo-as e as definições envolvidas na demonstração.
\\
Um exemplo é a proposição:
\begin{proposition}
Se $x$ é impar, $x^2$ é impar.
\end{proposition}
Sua demonstração é bem simples e fácil de ser lida:
\begin{demonstration}
Um número $x$ impar é por definição se $x=2a+1$, então $x^2 = (2a+1)^2 = 4a^2 + 4a +1 = 2(2a^2 + 2a) +1$.\\
Seja $b=2a^2 + 2a$, então $x^2 = 2b+1$, ou seja, um número impar. $\blacksquare$
\end{demonstration}
Nosso exercício aqui será analisar a demonstração acima.
\\~\\
Pimeiro chamamos a definição de um número impar, e repare que em nenhum momento ela é violada, fazemos uma álgebra simples para aparecer o que queremos, a definição de um número impar novamente.
\\~\\
Repare no $\blacksquare$ ao final da prova, a partir de agora o verá direto e é importante saber o que significa. Seu significado é abreviado como Q.E.D, que vem a expressão do latim "Quod erat demonstrandum", numa tradução livre, "como se queria demonstrar".
\\~\\
É importante que pegue algumas proposições simples como essa que provamos acima para que exercite essa habilidade, comece do básico para desenvolver o raciocínio e aos poucos aumente a dificuldade. Isso será inevitável em sua trilha pela matemática, com o tempo se verá insatisfeito com as provas de livros e conseguirá escrever as suas.
\section{Uso de casos}
Podermos usar os casos é de grande ajuda quando precisamos seguir um processo diferente caso um número seja negativo ou positivo, par ou impar, etc.
\\~\\
Pode ser que essas diferenças exijam diferentes tratamentos, mas é importante lembrar que é essencial que a generalidade da prova não seja perdida.
\\
Exemplo:
\begin{proposition}
Todo multiplo de 4 é igual à $1+(-1)^n(2n-1)$ para alguns $n \in \mathbb{N}$
\end{proposition}
\begin{demonstration}
\textbf{Caso 1:} se $n$ é par, então $(-1)^n =1$, um número $x$ é par se $x=2a$, portanto $1+(2(2a)-1) = 4a, \ \forall \ a$ \\
\textbf{Caso 2:} seja $n$ um número impar, logo $n=2a+1$ e $(-1)^n = -1$, portanto $1-2n+1] = -2n + 2= -2(2a+1) + 2 = -4a$
\\~\\
Portanto $1+(-1)^n(2n-1)$ é múltiplo de $4 \ \forall n$, seja ele par ou impar. $\blacksquare$ 
\end{demonstration}
\section{Tratamento de Casos Semelhantes}
Pode ser que hajam enganos e utilizemos duas provas diferentes de maneira desnecessária, isso deixa cansativo e tedioso de ser lida, temos que cuidar com os casos em que não existe essa necessidade.
\\
O exemplo que temos no livro é ótimo para vermos isso:

\begin{proposition}
Sejam $a, \ b  \in \mathbb{Z}$ com paridades opostos, então $a+b$ é impar.
\end{proposition}

Nesse caso poderíamos fazer no caso 1 com $a$ sendo par e $b$ impar, no caso 2 o contrário, mas ficaria massante de ler, portanto em respeito ao leitor omitirei essa parte.

\begin{demonstration}
Se $a = 2a$ e $b=2a+1$, então $a+b = 2a+2b+1 = 2(a+b) +1$, como $a,b \ \in \mathbb{Z} \ \Rightarrow a+b \ \in \mathbb{Z}$, $a+b$ é um número impar por definição. $\blacksquare$ 
\end{demonstration}
%%%%%%%%%%%%%%%%%%%%%%%% CHAPTER %%%%%%%%%%%%%%%%%%%%%%%%
\end{document}