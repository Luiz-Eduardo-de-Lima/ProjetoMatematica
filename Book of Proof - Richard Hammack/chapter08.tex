\documentclass[main.tex]{subfiles}
\begin{document}

Em seus estudos verá teoria dos conjuntos incontáveis vezes em diversas campos. Em especial a Análise Real (que é para o que estamos nos preparamos), todo livro que se preze tem um capítulo específico apenas para conjuntos. Isso nos revela o quão relevante e importante é o estudo dos conjuntos.
\par 

Conforme for avançando na matemática verá muitas provas envolvendo conjuntos, álgebra linear, abstrata, análise. Por isso é importante que aprender a ler e compreender provas envolvendo-os.
\par 

Caso seja necessário revise os primeiros capítulos do Projeto Matemática.

\section{Como provar que $ a \in A $}
Sabemos que um conjunto é determinado por uma regra $P(x)$, os elementos do conjunto universo que obedeçam essa $P(x)$ são elementos desse conjunto. Então, dados um conjunto qualquer:
$$A = \{x: P(x)\}$$

Para provar que um $x \in A$ temos que mostrar que $x$ satisfaz a condição $P(x)$.

\textbf{Exemplo:} Prove que $x \in A$ para $x = 5$ onde $A = \lbrace x \in \mathbb{N}: x > 4\rbrace$.
\textit{Prova:} Dados dois números $a$ e $b$, $a \geq b$ se, e somente se, $a - b > 0$. Supondo que $5 \leq 4$ então $5 - 4 \leq 0$, o que claramente é um absurdo. $\blacksquare$

\end{document}