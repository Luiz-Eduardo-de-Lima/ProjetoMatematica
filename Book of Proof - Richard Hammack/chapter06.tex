\documentclass[main.tex]{subfiles}
\begin{document}

\section{Provando Afirmações com Contradição}
A Prova por Contradição já foi mencionada nos capítulos anteriores e aprenderemos ela no decorrer desse capítulo.\\
Ela consistem assumir que o que quer provar seja falso leva à uma contradição, então o modelo é mais ou menos assim:
\textbf{Proposição }$P$\\
Assuma $\sim P$ \\
. \\
. \\
. \\
Então $C \land \sim C$
\\
Perceba que $C \land \sim C$ é a contradição que buscamos, pois algo não pode ser e não-ser ao mesmo tempo. Essa relação fica muito mais simples de ser compreendida numa tabela verdade. \\
\begin{center}
\begin{tabular}{|c | c || c | c | c || c |} \hline
$P$  & $C$ & $\sim P$ & $\sim C$ & $C \land \sim C$ & $(\sim P) \Rightarrow (C \land \sim C)$  \\ \hline 
V    &  V  &  F       &  F       &   F              &         T   \\
V    &  F  & F        &  V       &  F               &         T \\
F    &   V & V        &F         &  F               &       F \\
F    & F   & V        & V        & F                & F    \\ \hline 
\end{tabular}
\end{center}
Vamos fazer alguns exemplos para entender como funciona na prática:

\begin{definition}
Um número real $x$ é racional se $\exists \ a,b \in \mathbb{Z} \ tq. \ x = \frac{a}{b}$\\
Um número real $x$ é irracional se $\not\exists \ a,b \in \mathbb{Z} \ tq. \ x = \frac{a}{b}$
\end{definition}
\begin{proposition}
O número $\sqrt{6}$ é irracional
\end{proposition}
\textit{Prova:} Assuma que $\sqrt{6}$ é racional, então: 
$$\sqrt{6} = \frac{a}{b}$$
Vamos supor que $\frac{a}{b}$ esteja completamente reduzida (ou seja, seu \textbf{Máximo Divisor Comum} é 1), isso é, não temos $a$ e $b$ pares. Elevamos ambos os lados da equação ao quadrado:
$$6 = \frac{a^2}{b^2} \Rightarrow a^2 = 6b^2$$
Como podemos ver $a^2$ é múltiplo de 3, logo $a$ é múltiplo de 6 (Verifique), então pode ser escrito como $a = 6x$. \\
$$(6x)^2 = 6b^2) \Rightarrow 36x^2 = 6b^2) = 6x^2 = b^2$$
Então $b^2$ é múltiplo de $6$, logo $6 | b$. \\
Portanto $a$ e $b$ são múltiplos de $6$, o que viola nossa suposição, ou seja, uma contradição $\blacksquare$
\section{Provando Afirmações Condicionais por Contradição}
Nessa sessão trabalharemos com as afirmações condicionais, o processo é bem semelhante e novamente mostraremos uma tabela verdade para mostrar equivalência lógica, mas antes vamos ao modelo da prova: \\ \\
\textbf{Proposição} $P \Rightarrow Q$ \\
\textit{Prova:} Assuma $P \Rightarrow \sim Q$\\
. \\
. \\
. \\
($C \land \sim C$) \\ \\
Como já sabemos $\sim(P \Rightarrow Q)$ e $P \Rightarrow \sim Q$ são lógicamente equivalentes, mas mesmo assim vamos visualizar isso. \\
\begin{center}
\begin{tabular}{| c | c || c | c | c | c |} \hline
$P$ & $Q$ & $\sim Q$ & $P \Rightarrow Q$ & $\sim(P \Rightarrow Q)$ & $P \Rightarrow \sim Q$ \\ \hline
V & V & F & V & F & F \\ 
V & F & V & F & V & V \\ 
F & V & F & V & F & F \\ 
F & F & V & F & V & V \\ \hline
\end{tabular}
\end{center}
E novamente vamos por em prática:
\begin{proposition}
Dados $a,b,c \in \mathbb{Z}$, então $a^2 + 4b -2 \neq 0$
\end{proposition}
\textit{Prova:} Suponha que $a^2 + 4b -2 = 0$ isso implica que $a^2 + 4b = 2$, logo:
$$a = \sqrt{2-4b} \ \ \ (1)$$
$$b = \frac{2-a^2}{4} \ \ \ (2) $$
Substituindo (2) em $a^2 + 4b -2 = 0$:
$$a^2 + a^2 + 2 -2 = 0$$
\section{Combinando Técnicas}
\section{Uma Advertência}
As Provas por Contradição são realmente poderosas, porém o estudante deve sempre dar preferência por outras técnicas, essa é um último recurso caso uma Prova Direta ou Contrapositiva não tenha funcionado.

%%%%%%%%%%%%%%%%%%%%%%%%%%%%%%%%%%%%%%%%%%%%%%%%%%%
%                    PART                         %
%%%%%%%%%%%%%%%%%%%%%%%%%%%%%%%%%%%%%%%%%%%%%%%%%%%
\end{document}